\bfsection{\S 5. Hilbert零点定理と次元論初歩}


\bfsubsection{定理 5.6}
\barquo{
$r \geq \dim A$を示すには、$P$、$Q$が$k[X]=k[X_1, \cdots , X_n]$の素イデアルで$Q \supset P$、$Q \neq P$なら
\[
\trdeg_k k[X]/Q < \trdeg_k k[X]/P
\]
であることを示せば十分である。
}
\begin{rem}
先に$A = k[X]/P$としているが、この$P$は引用部の$P$とは別物である。
\end{rem}
\begin{proof}
  引用部を仮定する。$A$での素イデアルの鎖
  \[
  0=P_0 \subsetneq P_1 \subsetneq \cdots \subsetneq P_m
  \]
  を得る。よって引用部により
  \[
  \trdeg_k k[X]/\pi^{-1}(P_m) < \cdots < \trdeg_k k[X]/\pi^{-1}(P_1) < \trdeg_k A = r
  \]
  であるから$r \geq m$が判る。ゆえに、$\dim A \leq r$が結論される。
\end{proof}



\bfsubsection{定理 5.6}
\barquo{
$S = k[X_1, \cdots , X_r] - \single $とおくと$S$は乗法的集合で$P \cap S = \emptyset $、$Q \cap S = \emptyset$となる。
}
\begin{proof}
  $P \cap S \neq \emptyset$と仮定し、$f \in P \cap S$とする。このとき$\ol{f} \in k[X]/P$は$0$である。したがって$f(\gra_1, \cdots , \gra_r)=\ol{f(x_1, \cdots, x_r)} = 0$となる。ところが$\gra_1, \cdots , \gra_r$は$k$上代数的に独立だったから、これは矛盾。

  $Q \cap S = \emptyset$であることも同様。
\end{proof}



\bfsubsection{定理 5.6}
\barquo{
$P_{i}=Q_i \cap R$とおくと$P_i$は$S$と交わらない$R$の素イデアルで、したがって$\trdeg_k R/P_{r-1} > 0$、よって$P_{r-1}$は$R$の極大イデアルではない。
}
\begin{rem}
  $\trdeg_k R/P_{r-1} > 0$を示そう。$\ol{x}_1 \in R/P_{r-1}$が$k$上代数的でないことをいえばよい。$\ol{x}_1$が$k$上代数的であると仮定する。このとき、
  \[
  \sum_{i=0}^n a_i \ol{x}_1^i =0
  \]
  なる、$a_i \in k$がある。とくに$a_n \neq 0$となるようにとることができる。したがって
  \[
    \sum_{i=0}^n a_i x_1^i \in P_{r-1} \cap S = \emptyset
  \]
  となり矛盾である。
\end{rem}
